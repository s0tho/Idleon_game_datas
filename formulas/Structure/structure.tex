%%%%%%%%%%%%%%%%%%%%%%%%%%%%%%%%%%%%%%%%%
% Arsclassica Article
% Structure Specification File
%
% This file has been downloaded from:
% http://www.LaTeXTemplates.com
%
% Original author:
% Lorenzo Pantieri (http://www.lorenzopantieri.net) with extensive modifications by:
% Vel (vel@latextemplates.com)
%
% License:
% CC BY-NC-SA 3.0 (http://creativecommons.org/licenses/by-nc-sa/3.0/)
%
%%%%%%%%%%%%%%%%%%%%%%%%%%%%%%%%%%%%%%%%%

%----------------------------------------------------------------------------------------
%	REQUIRED PACKAGES
%----------------------------------------------------------------------------------------
\usepackage[english]{babel} % english 
\hbadness=99999  % or any number >=10000

\usepackage[
nochapters, % Turn off chapters since this is an article        
beramono, % Use the Bera Mono font for monospaced text (\texttt)
eulermath,% Use the Euler font for mathematics
pdfspacing, % Makes use of pdftex’ letter spacing capabilities via the microtype package
dottedtoc % Dotted lines leading to the page numbers in the table of contents
]{classicthesis} % The layout is based on the Classic Thesis style

\usepackage{arsclassica} % Modifies the Classic Thesis package

\usepackage[T1]{fontenc} % Use 8-bit encoding that has 256 glyphs
\usepackage{fancyhdr} % Header and footer
\usepackage{xcolor}
% \usepackage{sectsty}
\usepackage{etoolbox} 
\patchcmd{\chapter}{\thispagestyle{plain}}{\thispagestyle{fancy}}{}{}
\usepackage{lastpage}
\usepackage{titleps}
\usepackage{titlesec}

\usepackage[default,oldstyle,scale=0.95]{opensans}
\usepackage[defaultfam,tabular,lining]{montserrat}
\renewcommand{\familydefault}{\sfdefault}

\usepackage{csquotes}
\usepackage[utf8]{inputenc} % Required for including letters with accents

\usepackage{graphicx} % Required for including images
\graphicspath{{Figures/}} % Set the default folder for images

\usepackage{enumitem} % Required for manipulating the whitespace between and within lists

\usepackage{lipsum} % Used for inserting dummy 'Lorem ipsum' text into the template

\usepackage{subfig} % Required for creating figures with multiple parts (subfigures)

\usepackage{amsmath,amssymb,amsthm} % For including math equations, theorems, symbols, etc

\usepackage{varioref} % More descriptive referencing
\usepackage{changepage}
\newenvironment{subs}
  {\adjustwidth{3em}{0pt}}
  {\endadjustwidth}

\usepackage{multirow}
\usepackage{hyperref}
\usepackage{chngcntr}
\usepackage{amssymb}
\usepackage{multicol}
\usepackage{array}

\usepackage{eso-pic}
\usepackage{tocloft}

\counterwithin{figure}{section}
\counterwithin{table}{section}
\makeatletter
\@addtoreset{chapter}{part}
\makeatother 

% \usepackage{movie15}

\usepackage{geometry}
 \geometry{
 a4paper,
 total={170mm,250mm},
 left=20mm,
 top=15mm,
 }


%----------------------------------------------------------------------------------------
%	SECTION STYLE
%---------------------------------------------------------------------------------------
\definecolor{blue}{RGB}{0,51,153}

\titleformat{\chapter}{\fontfamily{opensans}\sffamily\huge\bfseries\color{blue}}{\thechapter}{20pt}{\huge}
\titleformat{\section}{\fontfamily{opensans}\sffamily\Large\bfseries\color{blue}}{\thesection}{1em}{}
\titleformat{\subsection}{\fontfamily{opensans}\sffamily\bfseries\color{blue}}{\hspace{1cm}}{0.5em}{\thesubsection\quad}

%----------------------------------------------------------------------------------------
%	CODING STYLE
%---------------------------------------------------------------------------------------

\usepackage{listings}
\usepackage{color}

\definecolor{mygray}{rgb}{0.5,0.5,0.5}

\lstdefinestyle{mystyle}{
keywordstyle=\color{magenta},
commentstyle=\color{mygray},
basicstyle=\ttfamily,
showstringspaces=false,
belowcaptionskip=1\baselineskip,
breaklines=true,
}
\lstset{style=mystyle}

%----------------------------------------------------------------------------------------
%	THEOREM STYLES
%---------------------------------------------------------------------------------------

\theoremstyle{definition} % Define theorem styles here based on the definition style (used for definitions and examples)
\newtheorem{definition}{Definition}

\theoremstyle{plain} % Define theorem styles here based on the plain style (used for theorems, lemmas, propositions)
\newtheorem{theorem}{Theorem}

\theoremstyle{remark} % Define theorem styles here based on the remark style (used for remarks and notes)


%----------------------------------------------------------------------------------------
%	HYPERLINKS
%---------------------------------------------------------------------------------------

\hypersetup{
%draft, % Uncomment to remove all links (useful for printing in black and white)
colorlinks=true, breaklinks=true,bookmarksnumbered,
urlcolor=webbrown, linkcolor=RoyalBlue, citecolor=webgreen, % Link colors
pdftitle={}, % PDF title
pdfauthor={\textcopyright}, % PDF Author
pdfsubject={}, % PDF Subject
pdfkeywords={}, % PDF Keywords
pdfcreator={pdfLaTeX}, % PDF Creator
pdfproducer={LaTeX with hyperref and ClassicThesis} % PDF producer
}


\renewcommand{\cftchapfont}{\Large\color{blue}}  
\renewcommand{\cftsecfont}{\color{blue}}

\pagestyle{fancy}
\fancyfoot[C]{}

\makeatletter
\renewcommand{\maketitle}{\bgroup\setlength{\parindent}{0pt}
\begin{flushleft}
  \textbf{\@title}
  ~\newline
  \newline
  \@date
  ~\newline
  \newline
  \@author
\end{flushleft}\egroup
}
\makeatother

\renewcommand{\headrulewidth}{0pt}
\renewcommand{\footrulewidth}{0pt}
\renewcommand{\chaptermark}[1]{\markboth{\MakeUppercase{#1}}{}}


